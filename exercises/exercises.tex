\documentclass{article}
\usepackage{amsmath}
\usepackage[utf8]{inputenc}
\usepackage{amsfonts}
\usepackage[english]{babel}
\usepackage[a4paper,bottom=1.5in,top=1.5in]{geometry}
\usepackage{fancyhdr}
\usepackage{answers}
\usepackage{fouriernc}
\usepackage[T1]{fontenc}
\usepackage{hyperref}
\hypersetup{
	colorlinks=true,pdfstartview={FitH},plainpages = false,linkcolor = black }

\parindent0mm
\parskip1.5ex plus0.5ex minus0.5ex
\pagestyle{fancy}
\lhead{\small Introduction to R}
\chead{}
\rhead{Exercises}
\rfoot{\thepage}
\cfoot{}

\begin{document}

\title{Introduction to R -- Exercises}
\author{Willi Mutschler \& Martina Danielova Zaharieva}
\date{Spring 2018}
\maketitle
\tableofcontents
\newpage
\setcounter{page}{1}
\section{Introduction}

\begin{enumerate}
\item Start R-Studio and have a look at all menu items.

%\item Read \textquotedblleft Hilfe -- Konsole\textquotedblright .

\item Under \textquotedblleft Tools - Options \ldots
\textquotedblright\ choose your preferred
setting.

\item Install the packages \texttt{MASS}, \texttt{foreign}, \texttt{xlsx}, 
\texttt{rgl}.

\item The current working directory (where R reads and writes files) can be
found by the command \texttt{getwd()}. Find your current working directory.

\item Use the  command \texttt{setwd("{}c:/path")} to change the working
directory to drive \texttt{c:} and path \texttt{/path}. Note that the path
name is structured by slashes (\texttt{/}), \emph{not} backslashes ($\backslash$). Change the working directory to \texttt{c:/temp}
and check if the change has been successful.\footnote{%
The working directory can also be changed via the menu: Tools -- Options \ldots}

%\item If you prefer to use an external editor (e.g. Tinn-R or Notepad++)
%install it. Make sure that it connects properly to R. You should be able to
%send single and multiple lines of code from the editor to R.

\item Open a new script file. Type the commands to perform the following
assignments:%
\begin{eqnarray*}
a &=&\frac{3\cdot (4+9)}{8-12.5} \\
b &=&\left( 1,4,1999,2011\right) \\
d &=&2\pi \\
e &=&a+d
\end{eqnarray*}%
Save the script and quit R.

\item Start R and re-open the script. Mark all lines (\textsc{Ctrl-A}) and
execute them (\textsc{Ctrl-R}). Print $a,b,d,e.$ Why is the variable name $c$
not used?

%\item If you use 
%TCIMACRO{\TeXButton{LaTeX}{\LaTeX}}%
%BeginExpansion
%\LaTeX%
%EndExpansion
%\ have a look at \textquotedblleft Hilfe -- Handb\"{u}cher (pdf) -- Sweave
%User\textquotedblright . Sweave allows to merge R code and R output (even
%graphics) with normal text.
\end{enumerate}
\newpage


\section{Logical operators}

\begin{enumerate}
\item Use the command \texttt{c()} to define the vectors%
\begin{eqnarray*}
x &=&\left( -1,0,1,4,9,2,1,4.5,1.1,-0.9\right) \\
y &=&\left( 1,1,2,2,3,3,4,4,5,NA\right) .
\end{eqnarray*}

\item Determine the lengths of the vectors using \texttt{length()}, check if 
\texttt{length(x)==length(y)}.

\item Perform the following logical operations:%
\begin{align*}
x &<&y \\
x &<&0 \\
x+3 &\geq &0 \\
y &<&0 \\
x<0 &\text{or}&y<0
\end{align*}

\item Use \texttt{all} to check if all elements of $x+3\geq 0.$

\item Use \texttt{all} to check if all elements of $y>0.$ Use \texttt{any}
to check if at least one element of $y>0.$
\end{enumerate}
\newpage

\section{Arithmetic operators and mathematical functions}

\begin{enumerate}
\item Define the vectors%
\begin{eqnarray*}
x &=&\left( -1,0,1,4,9,2,1,4.5,1.1,-0.9\right) \\
y &=&\left( 1,1,2,2,3,3,4,4,5,NA\right) .
\end{eqnarray*}%
and compute $x+y$ and $xy$ and $y/x$.

\item Compute $\ln (x)$. Determine the length of the result vector.

\item Use \texttt{any} to check if the vector \texttt{x} contains elements
satisfying $\sqrt{x}\geq 2$.

\item Compute%
\begin{eqnarray*}
a &=&\sum_{i=1}^{10}x_{i} \\
b &=&\sum_{i=1}^{10}y_{i}^{2}.
\end{eqnarray*}%
Use the \texttt{na.rm=TRUE} option (\textbf{na}-\textbf{r}e\textbf{m}ove) of
the \texttt{sum} command to drop the \texttt{NA} in $y$.

\item Compute%
\begin{equation*}
\sum_{i=1}^{10}x_{i}y_{i}^{2}.
\end{equation*}

\item The \texttt{sum} command is a convenient way to count the number of
elements satifying a certain condition. Count the number of elements of $x>0$%
.

\item Predict what the following commands will return:

\texttt{x\textasciicircum y}\newline
\texttt{x\textasciicircum(1/y)}\newline
\texttt{log(exp(y))}\newline
\texttt{y*c(-1,1)}\newline
\texttt{x+c(-1,0,1)}\newline
\texttt{sum(y*c(-1,1),na.rm=TRUE)}
\end{enumerate}
\newpage


\section{Matrix functions}

\begin{enumerate}
\item Define the matrix%
\begin{equation*}
X=\left[ 
\begin{array}{lll}
1 & 4 & 7 \\ 
2 & 5 & 8 \\ 
3 & 6 & 9%
\end{array}%
\right] ,
\end{equation*}%
print its transpose, its dimensions and its determinant.

\item Compute the trace of $X$ (i.e. the sum of its diagonal elements).

\item Type \texttt{diag(X) <- c(7,8,9)} to change the diagonal
elements. Compute the eigenvalues of (the new) $X$. %Is $X$ positive definite?

\item Invert $X$ and compute the eigenvalues of $X^{-1}$.

\item Define the vector $a=(1,3,2)$ and compute \texttt{a*X}, \texttt{a\%*\%X%
}, and \texttt{X\%*\%a}.

\item Compute the quadratic form $a^{\prime }Xa$.

\item Define the matrix%
\begin{equation*}
I=\left[ 
\begin{array}{lll}
1 & 0 & 0 \\ 
0 & 1 & 0 \\ 
0 & 0 & 1%
\end{array}%
\right]
\end{equation*}%
and define%
\begin{equation*}
Y=\left[ 
\begin{array}{lll}
1 & 4 & 7 \\ 
2 & 5 & 8 \\ 
3 & 6 & 9%
\end{array}%
\begin{array}{lll}
1 & 0 & 0 \\ 
0 & 1 & 0 \\ 
0 & 0 & 1%
\end{array}%
\right]
\end{equation*}%
and%
\begin{equation*}
Z=\left[ 
\begin{array}{c}
\begin{array}{lll}
1 & 4 & 7 \\ 
2 & 5 & 8 \\ 
3 & 6 & 9%
\end{array}
\\ 
\begin{array}{lll}
1 & 0 & 0 \\ 
0 & 1 & 0 \\ 
0 & 0 & 1%
\end{array}%
\end{array}%
\right]
\end{equation*}

\item Predict what the following commands will return:

\texttt{cbind(1,X)}\newline
\texttt{rbind(Y,c(1,2,3))}\newline
\texttt{X\%*\%I}\newline
\texttt{dim(X\%*\%Y)}\newline
\texttt{t(Y)+Z}\newline
\texttt{solve(t(Z)\%*\%Z)\%*\%(t(Z)\%*\%Z)}
\end{enumerate}
\newpage


\section{Set operations and special functions}

\begin{enumerate}
\item Define the vectors%
\begin{eqnarray*}
x &=&\left( -1,0,1,4,9,2,1,4.5,1.1,-0.9\right) \\
y &=&\left( 1,1,2,2,3,3,4,4,5,NA\right) .
\end{eqnarray*}%
and compute $x\cup y$. Determine the lengths of $x$, $y$ and $x\cup y$.

\item Count the number of elements of $y$ that are element of $x$.

\item Determine the length of the vector of unique elements of $y$.

\item Compute the vector $z$ with elements%
\begin{equation*}
z_{i}=\sum_{j=1}^{i}x_{j}
\end{equation*}%
for $i=1,\ldots ,10$.

\item Find the position of the largest
element of $x$.
\end{enumerate}
\newpage


\section{Sequences and replications}

\begin{enumerate}
\item Generate the vectors%
\begin{eqnarray*}
x_{1} &=&\left( 1,2,3,\ldots ,9\right) \\
x_{2} &=&\left( 0,1,0,1,0,1,0,1\right) \\
x_{3} &=&\left( 1,1,1,1,1,1,1,1\right) \\
x_{4} &=&\left( -1,1,-1,1,-1,1\right) \\
x_{5} &=&\left( 1980,1985,1990,\ldots ,2010\right) \\
x_{6} &=&\left( 0,0.01,0.02,\ldots ,0.99,1\right)
\end{eqnarray*}

\item Replications can also be generated for vectors of strings
(characters). Type

\texttt{a <- c("{}a","b","{}c")}\newline
\texttt{rep(a,3)}\newline
\texttt{rep(a,times=3)}\newline
\texttt{rep(a,each=3)}

\item Generate a grid of $n=500$ equidistant points on the interval $[-\pi
,\pi ]$.

\item Compare \texttt{1:10+1} and \texttt{1:(10+1)}.

\item Predict what the following commands will return:

\texttt{rep("bla",10)}\newline
\texttt{rep(rep(1:3,2),each=4)}\newline
\texttt{rep(c(1,6,NA,2),times=c(2,2,5,3))}
\end{enumerate}
\newpage


\section{The apply command}

Define the matrices%
\begin{eqnarray*}
X &=&\left[ 
\begin{array}{llll}
1 & 2 & 7 & 9 \\ 
9 & 5 & 6 & 4 \\ 
3 & 3 & 5 & 4%
\end{array}%
\right] , \\
Y &=&\left[ 
\begin{array}{lll}
1 & 4 & 7 \\ 
2 & 5 & NA \\ 
3 & 6 & 9%
\end{array}%
\right] .
\end{eqnarray*}%
In the following exercises always use the \texttt{apply} command.

\begin{enumerate}
\item Compute the row sums of $X$.

\item Compute the column means of $X$.

\item Compute the column ranges of $X.$

\item Compute the column sums of $Y$. Add the \texttt{sum} function option 
\texttt{na.rm=TRUE} to the \texttt{apply} command.

\item For each column of $X$, find the position (the row number) of the
largest element.

\item For each column of $X$, compute the cumulated sums (\texttt{cumsum}).

\item For each row of $X$, find the number of elements that are smaller than
5.

\item For each column of $Y$, determine the number of non-missing
observations.

\item For each row of $X$, determine the unique elements. Note that in this
case the \texttt{apply} command does not return a matrix.
\end{enumerate}
\newpage


\section{Reading and writing text files}

Download the files \texttt{bsp1.txt}, \texttt{bsp2.txt} and \texttt{bsp3.txt}
from the internet site of the course and save them to your working
directory. The three files contain computer generated random numbers.

\begin{enumerate}
\item Read the file \texttt{bsp1.txt} into a dataframe \texttt{x1}. Have a
look at the file and the data format \emph{before} you decide which reading
command you use (\texttt{read.csv}, \texttt{read.csv2} or \texttt{read.table}%
). Print the dataframe. If the dataframe is too large for your screen, you
can use the commands \texttt{head} and \texttt{tail} to print just parts of
it.

\item Read the files \texttt{bsp2.txt} and \texttt{bsp3.txt} into dataframes 
\texttt{x2} and \texttt{x3}. Note that \texttt{bsp2.txt} contains both
numeric and character entries. It is usually advisable to set the option 
\texttt{as.is=TRUE} when reading characters (strings).

\item Print the class of \texttt{x2}, its dimension, and its variable names
(use \texttt{names}).

\item Print a summary of \texttt{x3.}

\item Use the \texttt{apply} command to compute the mean and the standard
deviation of each column of \texttt{x3}.

\item Create a small dataframe \texttt{a} with two variables

%TCIMACRO{\TeXButton{BC}{\begin{center}}}%
%BeginExpansion
\begin{center}%
%EndExpansion
\begin{tabular}{cc}
x & y \\ \hline
1 & 4 \\ 
2 & 5 \\ 
3 & 6%
\end{tabular}%
%TCIMACRO{\TeXButton{EC}{\end{center}}}%
%BeginExpansion
\end{center}%
%EndExpansion

and write it to a file in your working directory.
\end{enumerate}
\newpage


\section{Reading data online}

Install and activate the packages \texttt{TTR} and \texttt{fImport}.

\begin{enumerate}
\item Read the (large) file \texttt{lest2001.csv} directly from the following internet
site into a dataframe \texttt{x}. The complete URL is

\texttt{http://www.wiwi.uni-muenster.de/05/download/}\newline
\texttt{studium/R/ws1112/data/lest2001.csv}

The file is the campus file of the German income tax records 2001 (the data
are provided by the Research Data Centre of the Federal Statistical Office,
they are described in \texttt{lest2001.pdf}). Take care to set the options
of the \texttt{read.csv} or \texttt{read.table} command correctly. The data
format is as follows:

\begin{itemize}
\item All data entries are separated by semi-colons.

\item The first row contains the variables names.

\item Missing values are denoted by a dot.

\item Apart from the last column all data are integer values.

\item The decimal sign in the last column is a dot.
\end{itemize}

Execute \texttt{y <- x\$zve}. The variable \texttt{y} now
contains the taxable income (\textbf{z}u \textbf{v}ersteuerndes \textbf{E}%
inkommen). Compute its range, its median, its mean, its variance, and the
0.01- and 0.99-quantiles. Remember to include the option \texttt{na.rm=TRUE}
in the functions.

\item Read the help text of the \texttt{getYahooData} command of the \texttt{%
TTR} package. Read the data about the Deutsche Bank (symbol \texttt{DB})
from January 1st, 2000 to the most recent date into the object \texttt{x}.
Plot the closing price by \texttt{plot(x\$Close)}.

%\item The \texttt{fImport} package provides easy access to the OANDA
%exchange rate internet site. Execute \texttt{x <-
%oandaSeries("{}EUR/USD")} to retrieve the daily exchange rate of the last
%366 days. Then plot \texttt{x}.
\end{enumerate}
\newpage


\section{Indexing vectors}

Define the following vectors%
\begin{equation*}
x=\left( 
\begin{array}{c}
1 \\ 
1.1 \\ 
9 \\ 
8 \\ 
1 \\ 
4 \\ 
4 \\ 
1%
\end{array}%
\right) ,\quad y=\left( 
\begin{array}{c}
1 \\ 
2 \\ 
3 \\ 
4 \\ 
4 \\ 
3 \\ 
2 \\ 
NA%
\end{array}%
\right) ,\quad z=\left( 
\begin{array}{c}
TRUE \\ 
TRUE \\ 
FALSE \\ 
FALSE \\ 
TRUE \\ 
FALSE \\ 
FALSE \\ 
FALSE%
\end{array}%
\right)
\end{equation*}

\begin{enumerate}
\item Predict what the following commands will return (and then check if you
are right):

\texttt{x[-2]}\newline
\texttt{x[2:5]}\newline
\texttt{x[c(1,5,8)]}\newline
\texttt{x[-c(1,5,8)]}\newline
\texttt{x[y]}\newline
\texttt{x[seq(2,8,by=2)]}\newline
\texttt{x[rep(1:3,4)]}

\item Predict what the following commands will return (and then check if you
are right):

\texttt{y[z]}\newline
\texttt{y[!z]}\newline
\texttt{y[x>2]}\newline
\texttt{y[x==1]}\newline
\texttt{x[!is.na(y)]}\newline
\texttt{y[!is.na(y)]}

\item Indexing is not only used to read certain elements of a vector but
also to change them. Execute \texttt{x2 <- x} to make a copy of 
\texttt{x}. Change all elements of \texttt{x2} that have the value 4 to the
value $-4$. Print \texttt{x2}.

\item Change all elements of \texttt{x2} that have the value 1 to missing
value (\texttt{NA}). Print \texttt{x2}.

\item Execute \texttt{x2[z] <- 0}. Print \texttt{x2}.
\end{enumerate}
\newpage


\section{Indexing matrices}

Define the matrix \texttt{x <- matrix(c(1:12,12:1),4,6)}.

\begin{enumerate}
\item Predict what the following commands will return (and then check if you
are right):

\texttt{x[1,3]}\newline
\texttt{x[,5]}\newline
\texttt{x[2,]}\newline
\texttt{x[,-3]}\newline
\texttt{x[-4,]}\newline
\texttt{x[2:3,3:4]}\newline
\texttt{x[2:4,4]}

\item Predict what the following commands will return (and then check if you
are right):

\texttt{x[x>5]}\newline
\texttt{x[,x[1,]<=5]}\newline
\texttt{x[x[,2]>6,]}\newline
\texttt{x[x[,2]>6,4:6]}\newline
\texttt{x[x[,1]<3 \& x[,2]<6,]}

\item Print all rows where column 5 is at least three times larger than
column 6.

\item Count the number of elements of \texttt{x} that are larger than 7.

\item Count the number of elements in row 2 that are smaller than their
neighbours in row 1.

\item Count the number of elements of \texttt{x} that are larger than their
left neighbour.
\end{enumerate}
\newpage


\section{Indexing dataframes}

Load the dataset \texttt{bsp2.txt} as dataframe \texttt{y} and print it.

\begin{enumerate}
\item Use different ways to print the second column of the dataframe \texttt{%
y} (as a vector or a data\-frame).

\item Use different ways to print columns $U$ and $V$.

\item Use the \texttt{attach} command to make the variables directly
accessible. Print \texttt{X}. Now \texttt{detach} the dataframe again.

\item Print all rows of \texttt{y} where the variable $U$ has value A or B.

\item Print all rows of \texttt{y} where the variable $X$ is smaller than
its median and the variable $Y$ is larger than its median.

\item One can add row names to a dataframe. Execute the following command:

\texttt{row.names(y) <-
paste(rep(LETTERS[1:20],each=2),rep(1:2,20),sep={}"}{}\texttt{")}

and print the dataframe to have a look at the new row names.

\item Use the row name and the variable name to print the value of variable 
\texttt{Z} at observation \texttt{T1}.

\item Print the rows for observations \texttt{G1} and \texttt{G2}.
\end{enumerate}

\newpage


\section{User-defined functions}

\begin{enumerate}
\item Define a Cobb-Douglas production function with two inputs vectors,%
\begin{eqnarray*}
x &=&\left( 
\begin{array}{c}
L \\ 
K%
\end{array}%
\right) \\
\theta &=&\left( 
\begin{array}{c}
A \\ 
\alpha \\ 
\beta%
\end{array}%
\right)
\end{eqnarray*}%
and scalar output%
\begin{equation*}
y=AL^{\alpha }K^{\beta }.
\end{equation*}%
Evaluate the function at%
\begin{eqnarray*}
x &=&\left( 
\begin{array}{c}
2 \\ 
3%
\end{array}%
\right) \\
\theta &=&\left( 
\begin{array}{c}
1 \\ 
0.3 \\ 
0.8%
\end{array}%
\right) .
\end{eqnarray*}

\item Define a function \texttt{lowdecile} with one input vector $\left(
x_{1},\ldots ,x_{n}\right) $ of arbitrary length. The function should
compute and return the mean of all observations in the lowest decile. Define
the vector%
\begin{equation*}
x=\left( 0,0,0,0,1,1,1,1,2,2,2,2,\ldots ,9,9,9,9\right)
\end{equation*}%
and apply \texttt{lowdecile} to $x$.
\end{enumerate}


\newpage



\section{Sorting and merging}

\begin{enumerate}
\item Define $x=\left( -1,0,1,4,9,2,1,4.5,1.1,-0.9\right) $ and sort the
vector ascendingly. Print the second smallest element of $x$.

\item Sort $x$ descendingly and print the third largest element of $x.$

\item Execute \texttt{x <- matrix(c(1:12,12:1),6,4,byrow=T)} to
define the matrix $x.$ Calculate the order vector $p$ of the last column of $%
x.$ Sort the matrix $x$ by the last column and print the sorted matrix.

\item Load the file \texttt{bsp2.txt} into the dataframe \texttt{x}. Sort 
\texttt{x} by the variable \texttt{U} and on the next level by the variable 
\texttt{X}.

\item Activate the package \texttt{foreign} and load the Stata files \texttt{%
wave2000.dta} and \texttt{wave2009.dta} into two dataframes. Print the
variable names of both dataframes. Merge the dataframes by the variable 
\texttt{pid}. Keep only persons who are present in both years (by setting
the option \texttt{all=FALSE}). Print the variables names of the merged
dataframe.

\item Calculate the mean life satisfaction in 2000 and 2009.

\item Calculate the mean change in life satisfaction for persons who were
single in 2000 and married in 2009.
\end{enumerate}
\newpage


\section{Frequency tables}

\begin{enumerate}
\item Read the Stata file \texttt{wave2009.dta} into a dataframe. Tabulate
the variables \texttt{gender}, \texttt{marital} and \texttt{children}.

\item Read the file \texttt{bsp2.txt} into a dataframe. Compute and plot the
frequency tables of the variables $U$ and $V.$

\item Define the vector $x=(1,1,1,1,1,2,2,2,2,3,3,3,4,4,5,9)$. Compute the
absolute and relative frequency tables and determine the number of different
values of $x.$

\item Read the file \texttt{lest2001.csv} (wage and income tax data) into a
dataframe using the option \texttt{as.is=TRUE}. The variable \texttt{ef8}
gives the gender (male=0, female=1). Tabulate \texttt{ef8}.

\item The \texttt{lest2001.csv} dataset contains a weighting variable (\texttt{%
samplingweight}). Each row counts as \texttt{samplingweight} observations.
Re-compute the absolute frequencies of \texttt{ef8} taking into account the
weights.

\item Load the Stata file \texttt{mikrozensus2002cf.dta} into a dataframe
(the data are described in the file \texttt{mikrozensus2002cf.pdf}).
Consider the variable \texttt{ef455} (age of flat or house) which is coded
into nine classes. Tabulate the age structure.
\end{enumerate}
\newpage


\section{Cumulative distribution function and quantile function}

\begin{enumerate}
\item Load the dataset \texttt{bsp1.txt} and calculate the value of the
empirical distribution function of \texttt{Variable1} at the point $x=10.$

\item Plot the empirical distribution function of \texttt{Variable2}.
Improve the readability of the plot by adding an appropriate heading and
axes labels (use the options \texttt{main} and \texttt{xlab,ylab}).

\item Compute the $p$-quantiles of \texttt{Variable1} for $%
p=0.01,0.05,0.1,0.5,0.9,0.95,0.99$.

\item Load the dataset \texttt{lest2001.csv} and compute the value of the
empirical distribution function of \texttt{zve} (taxable income) at the
points 0, 12000, and 60000.

\item Use the \texttt{quantile} command to compute the 0.99-quantile, the
0.999-quantile, and the 0.9999-quantile of \texttt{zve}.

%\item Redo the last two exercises taking into account the different sampling
%weights (given in the variable \texttt{samplingweight}).

%\item Define the vector $x=(1,1,1,1,1,2,2,2,2,3,3,3,4,4,5)^{\prime }$ and
%plot its quantile function at all percentiles (i.e. at 0.01, 0.02, \ldots ,
%0.99). Change the \texttt{type} option of the quantile command to see its
%impact.
\end{enumerate}
\newpage


\section{Mean, variance and standard deviation}

\begin{enumerate}
\item Read the file \texttt{lest2001.csv} (wage and income tax data) into a
dataframe. The variable \texttt{zve} reports taxable income. Compute the
mean taxable income without weighting the observations. Then re-compute the
mean using the weights given in the \texttt{samplingweight}. Hint: Use 
\texttt{weighted.mean}.

\item Compute the unweighted and weighted variance of \texttt{zve}.

\item Load the Stata file \texttt{mikrozensus2002cf.dta} into a dataframe
and consider the variables \texttt{ef455} (age of flat or house) and \texttt{%
ef466} (cost of heating and warm water in April 2002). Compute the mean cost
of heating and warm water for each age class. Hint: You may consider using
the command \texttt{by}.
\end{enumerate}
\newpage


\section{Histograms}

In this section, please always use the command \texttt{truehist} (which is
included in the \texttt{MASS} package) to generate histograms.

\begin{enumerate}
\item Load the file \texttt{gemeinden2006.csv} into a dataframe. Delete all
observations where the number of inhabitants (\texttt{Einwohner}) is smaller
than 5. Plot the histogram of the logarithm of the variable \texttt{Einwohner%
}.

\item Add the density function of a fitted normal distribution to the
%histogram.

\item Load the Stata file \texttt{mikrozensus2002cf.dta} into a dataframe.
Consider the variable \texttt{ef462} (rent in April 2002). Drop all
observations where the rent exceeds 2000 Euro. Plot the histogram

\item Load the Stata file \texttt{mikrozensus2002cf.dta} into a dataframe.

\begin{enumerate}
\item Plot the histogram of the variable \texttt{ef453} (size of flat in
square meters).

\item Drop all observations with more than $300$ $m^{2}$ and plot the
histogram again.

\item Set the number of bins in the histogram to 15.
\end{enumerate}
\end{enumerate}
\newpage


\section{Contingency tables, correlation and covariance}

\begin{enumerate}
\item Execute \texttt{data(Titanic)} to load the object \texttt{Titanic} of
class \texttt{table}. Print it as an ordinary table and as a flat table.
Plot it as well. Compute the univariate marginal distributions using the 
\texttt{apply} command. Compute the bivariate marginal distribution of
survival and social class (again using \texttt{apply}).

\item Load the file \texttt{covmat.csv} into a dataframe.

\begin{enumerate}
\item Compute the covariance matrix using the option \texttt{use="{}complete"%
}. Check if the covariance matrix is positive definite.

\item Now compute the covariance using the option \texttt{"{}pairwise"} and
check again, if the covariance matrix is positive definite.
\end{enumerate}

\item Load the Stata file \texttt{mikrozensus2002cf.dta} into a dataframe.
Consider the two variables \texttt{ef141} (normal hours worked) and \texttt{%
ef372} (net income per capita in April 2002). The variable \texttt{ef372} is
coded into 24 income brackets (see data description \texttt{%
mikrozensus2002cf.pdf}). Drop all observations where \texttt{ef372} is
larger than 24 (i.e. missing values etc.) and recode the remaining
observations to the midpoints of the income brackets. Compute the
correlation coefficient of hours worked and net income.
\end{enumerate}

\newpage


\section{Programming}

\begin{enumerate}
\item This exercise illustrates that loops are often not very efficient.

\begin{itemize}
\item Create the vector $x=(1,2,\ldots ,1\,000\,000)$ and convert it from 
\texttt{integer} to \texttt{numeric} using the conversion command \texttt{%
as.numeric}.

\item Set $S=0$.

\item Write a \texttt{for}-loop to compute the sum of all vector elements
without using the \texttt{sum} command.

\item Put the command \texttt{p0 <- proc.time()[3]} in front of
the loop and the command \texttt{print(proc.time()[3]-p0)} at the end. These
commands allow to measure the execution time of the loop.

\item Compare your result with the execution time of the \texttt{sum}
command.
\end{itemize}

\item Create a grid vector $x$ of 60 equidistant points $x_{1},\ldots
,x_{60} $ on the interval $[-10,10]$, and another grid vector $y$ of 70
points $y_{1},\ldots ,y_{70}$ on $[-10,10]$. Create an empty matrix $Z$ of
dimension $60\times 70$.

Write a double loop to compute the matrix elements%
\begin{equation*}
Z_{ij}=\frac{10}{r_{ij}}\cdot \sin (r_{ij})
\end{equation*}%
where $r_{ij}=\sqrt{x_{i}^{2}+y_{j}^{2}}$. Execute \texttt{persp(x,y,Z)}.

\item Load the dataset \texttt{fussballdaten.csv}. It contains all
\textquotedblleft 1. Bundesliga\textquotedblright\ results between the
saisons 1996/1997 and 2008/2009.

\begin{itemize}
\item Create an alphabetically ordered vector of all clubs in the dataset.

\item Write a loop over all clubs. For each club compute the proportion of
games won.

\item Order the clubs descendingly according to the proportion of games won
and plot a \texttt{barplot} of the proportion.
\end{itemize}

\item Load the dataset \texttt{bsp1.txt}. Use the command \texttt{ifelse} to
change all zero entries of the first variable to ones (and leave all other
entries unchanged).
\end{enumerate}
\newpage


\section{Random numbers}

This section is not only about random number generation but also includes
exercises about the R-functions for standard distributions in statistics.

\begin{enumerate}
\item Let $X\sim N\left( 0,1\right) $. Compute the probability $P(|X|>3.5)$.
Generate $n=10000$ random draws $X_{1},\ldots ,X_{n}$ from $X$ and count the
number of observations $|X_{i}|>3.5$. Repeat drawing random samples $R=5000$
times and write the counts into a vector $Z_{1},\ldots ,Z_{5000}$ of length
5000. Tabulate $Z$ and compare the frequencies with the probability function
of a suitably fitted Poisson distribution.

\item Generate $n=10000$ draws from a lognormal distribution $X\sim e^{Y}$
where $Y\sim N(1,0.5^{2})$ (the parameters in the R function are \texttt{%
meanlog=1} and \texttt{sdlog=0.5}). Split the screen into two plotting areas
using the commnd \texttt{par(mfrow=c(2,1))}. Plot the histograms of $X$ and $%
\ln X$.

\item Generate $n=10000$ draws from $X\sim N(0,1)$. Compute the cumulated
means, i.e.%
\begin{equation*}
\bar{X}_{j}=\frac{1}{j}\sum_{i=1}^{j}X_{i}
\end{equation*}%
for $j=1,\ldots ,n$ and plot them. Hint: Use the command \texttt{cumsum}.

%\item Generate a sample $X_{1},\ldots ,X_{n}$ of size $n=10000$ from the
%Cauchy distribution (that is a $t$-distribution with one degree of freedom).
%Compute the cumulated means, i.e.%
%\begin{equation*}
%\bar{X}_{j}=\frac{1}{j}\sum_{i=1}^{j}X_{i}
%\end{equation*}%
%for $j=1,\ldots ,n$ and plot them. Use the option \texttt{ylim} to restrict
%the y-axis to the range $[-5,5]$. Repeat the sampling several times.
%
%\item To generate random numbers with distributions not implented in R, one
%can use the inversion method. Let $X$ be a continuous random variable with
%cdf $F_{X}$ and quantile function $F_{X}^{-1}$. If $U$ is uniformly
%distributed on $[0,1]$, then $F_{X}^{-1}(U)$ has cdf $F_{X}$. The cdf of the
%Pareto distribution with parameters $K$ and $\alpha $ is 
%\begin{equation*}
%F_{X}(x)=1-Kx^{-\alpha }.
%\end{equation*}
%
%\begin{enumerate}
%\item Let $K=1$. Derive the quantile function and generate $n=10000$ draws $%
%X_{1},\ldots ,X_{n}$ from the Pareto distribution with parameter $\alpha
%=1.3 $.
%
%\item Plot the histogram of $X_{1},\ldots ,X_{n}$.
%
%\item Plot the histogram of $\ln X_{1},\ldots ,\ln X_{n}$.
%
%\item Add the density function of a fitted exponential distribution to the
%histogram of $\ln X_{1},\ldots ,\ln X_{n}$. Hint: Set the parameter \texttt{%
%rate} of the fitted exponential distribution to the inverse of the mean of $%
%\ln X_{1},\ldots ,\ln X_{n}$.
%\end{enumerate}
\end{enumerate}
\newpage


\section{Simulations}

\begin{enumerate}
\item This exercise illustrates the one-sample $t$-test.

\begin{enumerate}
\item Generate $n=10$ obserations from $X\sim N(10,3^{2})$. Compute the mean
and the standard deviation of $X_{1},\ldots ,X_{10}$.

\item The $t$-statistics of the hypothesis test $H_{0}:\mu =10$ against $%
H_{1}:\mu \neq 10$ is%
\begin{equation*}
t=\sqrt{10}\frac{\bar{X}-10}{sd}
\end{equation*}%
where $sd$ is the standard deviation (as computed by \texttt{sd}). Compute
the $t$-statistic.

\item Create an empty vector $Z$ of length $R=5000$. Write a loop over $%
r=1,\ldots ,R$ and repeat steps (a) and (b) for each $r$. Save the $t$%
-statistic at $Z_{r}$.

\item Plot the histogram of $Z_{1},\ldots ,Z_{R}$ and add the density
function of the $t_{9}$-distribution.
\end{enumerate}

\item The classical central limit theorem states that the standardized sum
of i.i.d. random variables with finite variance converges in distribution to
the standard normal distribution $N(0,1)$. This exercise illustrates the
central limit theorem.

\begin{enumerate}
\item Write a simulation that performs the following steps:

\begin{itemize}
\item Generate a random sample $X_{1},\ldots ,X_{5}$ of size $n=5$ from the
standard exponential distribution $Exp(1)$.

\item Compute the sample sum.

\item Repeat the steps $R=10\,000$ times. For each replication, store the
sum, e.g. into a vector $Z$.

\item Plot the histogram of the sum and add the density function of $%
N(m,s^{2})$ where $m$ is the mean of $Z$ and $s$ is the standard deviation
of $Z$.
\end{itemize}

\item Increase the sample size $n$ in (a) to $n=50,500,5000$.

\item Redo (a) and (b) with other distributions than the exponential. Use
the uniform distribution, the $t$-distribution with 3 degrees of freedom,
the Bernoulli distribution (i.e. binomial with parameter \texttt{size=1}),
the Poisson distribution.

\item The central limit theorem breaks down if the variance of the summands
is infinite. Redo (a) and (b) using a $t$-distribution with only 1.5 degrees
of freedom.
\end{enumerate}

%\item This exercise illustrates convergence of random sums to the Laplace
%distribution.
%
%\begin{enumerate}
%\item Write a simulation that performs the following steps:
%
%\begin{itemize}
%\item Generate a random number $\nu $ from the geometric distribution with
%support $\{1,2,\ldots \}$ and parameter $p=0.02$. Use the command \texttt{%
%rgeom(n=1,prob=0.02)+1}. (If you do not add one the sample size could be
%zero.)
%
%\item Generate a random sample $X_{1},\ldots ,X_{\nu }$ of random size $\nu $
%from the centralized standard exponential distribution $Exp(1)-1$, i.e.
%substract 1 from the exponentially distributed random numbers.
%
%\item Compute the sample sum and standardize it by multiplying it with $%
%\sqrt{2p}$.
%
%\item Repeat the steps $R=10000$ times. For each replication, store the
%standardized sum, e.g. into a vector $Z$.
%
%\item Plot the histogram of the standardized sum and add the density
%function of the standard Laplace distribution,%
%\begin{equation*}
%f(x)=\frac{1}{2}e^{-|x|}.
%\end{equation*}
%\end{itemize}
%
%\item Increase the expected sample size $1/p$ in (a) with $p=0.002,0.0002.$
%
%\item Redo (a) and (b) replacing the shifted exponential distribution by the
%uniform distribution on $[-\sqrt{3},\sqrt{3}]$, and by the standard normal
%distribution $N(0,1)$.
%\end{enumerate}
\end{enumerate}
\newpage


\section{Linear regressions}

\begin{enumerate}
\item Load the Stata dataset \texttt{wages.dta}. The variables are \texttt{%
earnings} (in Euro, 2009), \texttt{age}, \texttt{gender} (male=1, female=2), 
\texttt{education} (year of education), \texttt{hours} (hours worked during
2009), and \texttt{weight}.

\begin{enumerate}
\item Compute the (unweighted) wage equation%
\begin{equation*}
\ln \text{earnings}_{i}=\alpha +\beta _{1}\text{age}_{i}+\beta _{2}\text{age}%
_{i}^{2}+\beta _{3}\text{education}_{i}+\beta _{4}\text{gender}_{i}+u_{i},
\end{equation*}%
print the summary of the \texttt{lm}-object, and interpret the output.

\item Add an interaction term for \texttt{education} and \texttt{gender} to
the regression.

\item Compute the weighted hourly wage equation%
\begin{equation*}
\ln \frac{\text{earnings}_{i}}{\text{hours}_{i}}=\alpha +\beta _{1}\text{age}%
_{i}+\beta _{2}\text{age}_{i}^{2}+\beta _{3}\text{education}_{i}+\beta _{4}%
\text{gender}_{i}+u_{i},
\end{equation*}%
print the summary of the \texttt{lm}-object, and interpret the output.

\item Activate the package \texttt{sandwich} (or the package \texttt{AER}).
Use the function \texttt{coeftest} to compute the heteroskedasticity robust
standard errors for the estimated coefficients.

\item Predict the hourly wage of a male person aged 60 years as a function
of education (vary the years of education between 9 and 18). Set the option 
\texttt{se.fit=TRUE}. Inspect the object returned by the \texttt{predict}
command. Plot the predicted values and add the $\pm 2$ standard deviations
confidence intervals.
\end{enumerate}

\item Load the dataset \texttt{bsp4.txt}.

\begin{enumerate}
\item Plot the scatterplot of $y$ against $x$.

\item Perform a simple linear regression of $y$ on $x$ and save the results
as an \texttt{lm}-object \texttt{obj}. Add the regression line of $y$ on $x$
to the plot.

\item Extract the fitted values from \texttt{obj} and add them as red points
to the plot (use the command \texttt{points}).

\item Extract the residuals of the regression and calculate the sum of the
squared residuals,%
\begin{equation*}
SSR=\sum_{i=1}^{100}\hat{u}_{i}^{2}
\end{equation*}

\item Compute the total sum of squares and the explained sum of squares,%
\begin{eqnarray*}
TSS &=&\sum_{i=1}^{100}\left( y_{i}-\bar{y}\right) ^{2} \\
ESS &=&\sum_{i=1}^{100}\left( \hat{y}_{i}-\bar{y}\right) ^{2}
\end{eqnarray*}%
and show that $ESS+SSR=TSS.$
\end{enumerate}

%\item Regression results can be severely misleading if a relevant exogenous
%variable is missing in the regression model (omitted variable bias).
%Consider the true regression relation%
%\begin{equation*}
%y_{i}=\alpha +\beta _{1}x_{1i}+\beta _{2}x_{2i}+u_{i}
%\end{equation*}%
%with $\alpha =3$, $\beta _{1}=1$, $\beta _{2}=2$, and $u_{i}\sim N(0,\sigma
%^{2})$ where $\sigma ^{2}=4$. We will investigate what happens if the
%variable $x_{2}$ is omitted. Write a simulation that performs the following
%steps:
%
%\begin{itemize}
%\item Create an empty matrix $Z$ of dimensions $R\times 2$ with $R=10\,000$.
%
%\item Generate a bivariate random sample 
%\begin{equation*}
%\left( 
%\begin{array}{c}
%x_{11} \\ 
%x_{12}%
%\end{array}%
%\right) ,\ldots ,\left( 
%\begin{array}{c}
%x_{n1} \\ 
%x_{n2}%
%\end{array}%
%\right)
%\end{equation*}%
%of size $n=100$ from the multivariate normal distribution with parameters%
%\begin{equation*}
%\mu =\left( 
%\begin{array}{c}
%1 \\ 
%1%
%\end{array}%
%\right) ,\quad \Sigma =\left( 
%\begin{array}{ll}
%1 & 0.7 \\ 
%0.7 & 1%
%\end{array}%
%\right) .
%\end{equation*}%
%The easiest way to do so is to use the command \texttt{mvrnorm} supplied by
%the \texttt{MASS} package, e.g. \texttt{x <-
%mvrnorm(n=100,mu=c(1,1),Sigma=matrix(c(1,0.7,0.7,1),2,2))}. The return
%matrix \texttt{x} has two columns, the first column is variable $%
%x_{1}=(x_{11},\ldots ,x_{n1})$, the second column is $x_{2}=(x_{12},\ldots
%,x_{n2})$.
%
%\item Generate a random sample $u_{1},\ldots ,u_{n}$ of size $n=100$ from $%
%N(0,\sigma ^{2})$ with $\sigma ^{2}=4$.
%
%\item Compute the endogenous variables $y_{i}=\alpha +\beta _{1}x_{1i}+\beta
%_{2}x_{2i}+u_{i}$.
%
%\item Perform a simple linear regression of $y$ on $x_{1}$, omitting the
%relevant variable $x_{2}$.
%
%\item Write a loop over $r=1,\ldots ,R$. For each replication, store the
%estimated coefficients $\hat{\alpha}$ and $\hat{\beta}_{1}$ is row $%
%r=1,\ldots ,R$ of $Z$.
%
%\item Split the screen using \texttt{par(mfrow=c(2,1))}. Plot the histograms
%of the distribution of $\hat{\alpha}$ (the first column of $Z$) and of the
%distribution of $\hat{\beta}_{1}$ (second column of $Z$). Add the true
%values $\alpha =3$ and $\beta _{1}=1$ as vertical red lines to the plot
%(using \texttt{abline(v=...)}).
%\end{itemize}
\end{enumerate}
\newpage


\section{Numerical optimization}

\begin{enumerate}
\item Define the polynomial \texttt{function}%
\begin{equation*}
f(x)=x\left( x-2\right) \left( x-4\right) \left( x-5\right) -10
\end{equation*}%
and plot it over the interval $[0,5]$.

\begin{enumerate}
\item Use \texttt{optimize} to find the minimum of the function. Set the
interval to $[0,5]$ and then to $[0,6].$

\item Even though one-dimensional optimization problems should be solved by 
\texttt{optimize}, use the \texttt{optim}-command with option \texttt{%
method="BFGS"} to find the minimum of the function $f$. Set the starting
values to $0,1,2,3,-1$ and check if the algorithm finds the global minimum.
\end{enumerate}

\item Consider the Cobb-Douglas production function with two inputs $x_{1}$
and $x_{2}$,%
\begin{equation*}
y=f(x_{1},x_{2})=x_{1}^{0.3}x_{2}^{0.6}.
\end{equation*}%
Suppose the output price is $p_{y}=1$ and the input prices are $p_{1}=2$ and 
$p_{2}=0.5$. Define the profit function as a \texttt{function} and
numerically maximize it.

\item Consider the nonlinear regression model%
\begin{equation*}
y_{i}=\exp \left( \alpha +\beta x_{i}\right) +u_{i}
\end{equation*}%
where $u_{i}\sim N(0,\sigma ^{2})$. Since the error term is additive one
cannot simply take logarithms to make the model linear. Load the dataset 
\texttt{expgrowth.csv} and estimate the parameters $\alpha $ and $\beta $ by
minimizing%
\begin{equation*}
\sum_{i=1}^{n}\left( y_{i}-\exp \left( a+bx_{i}\right) \right) ^{2}
\end{equation*}%
numerically with respect to $a$ and $b$.
\end{enumerate}
\newpage


\section{Maximum likelihood}

\begin{enumerate}
\item Load the dataset \texttt{round86.csv} into a dataframe. The first
column is the actual earnings in 1986 of $n=443$ persons as derived from the
employers files, the second column is the earnings reported by the persons.

\begin{enumerate}
\item Compute the reporting errors and plot their histogram.

\item Assume that the reporting error $x$ follows a Laplace distribution
with density%
\begin{equation*}
f\left( x\right) =\frac{1}{2b}\exp \left( -\frac{|x-\mu |}{b}\right) .
\end{equation*}%
Write a function of $\theta =(\mu ,b)$ and $x=(x_{1},\ldots ,x_{n})$ to
compute the log-likelihood function%
\begin{equation*}
\ln L\left( \theta ,x\right) =\sum_{i=1}^{n}\ln f\left( x\right) .
\end{equation*}%
Hint: The absolut value function $|\cdot |$ is \texttt{abs(}$\cdot $\texttt{)%
}.

\item Numerically maximize the log-likelihood function with respect to $\mu $
and $b$. Set the starting values to $\mu _{0}=0$ and $b_{0}=3000$.
\end{enumerate}

\item Load the dataset \texttt{fussballdaten.csv} into a dataframe.

\begin{enumerate}
\item Use indexing to obtain the vector of goals scored in home matches by
the team \texttt{dortmund}. Tabulate and plot the vector.

\item Fit a Poisson distribution to the number of goals. The maximum
likelihood estimator of the parameter $\lambda $ of the Poisson distribution
is $\hat{\lambda}=\bar{X}$ where $\bar{X}$ is the mean number of goals. Add
the probabilities of the fitted Poisson distribution to the plot (as red
points).

%\item Re-do the exercise for other teams.
\end{enumerate}

%\item Income distributions are sometimes approximated by a log-normal
%distribution with density function%
%\begin{equation*}
%f\left( x\right) =\frac{1}{\sqrt{2\pi }\sigma x}\exp \left( -\frac{1}{2}%
%\left( \frac{\ln x-\mu }{\sigma }\right) ^{2}\right)
%\end{equation*}%
%where $\mu \in \mathbb{R}$ and $\sigma >0$ are parameters. Load the Stata
%dataset \texttt{incomes.dta}. The variable $x$ reports the annual household
%income after taxes and benefits of $n=27204$ German households in 2009.
%
%\begin{enumerate}
%\item Write an R function with arguments $\theta =(\mu ,\sigma )$ and $%
%x=(x_{1},\ldots ,x_{n})$ to compute the log-likelihood of the log-normal
%distribution. The log-normal density can be calculated in R by the command 
%\texttt{dlnorm} (this is the density of a log-normal distribution, not the
%log-density of a normal distribution).
%
%\item Fit the log-normal distribution to the data. Use the start values $\mu
%_{0}=10$ and $\sigma _{0}=0.5$. Restrict the parameter space to $\sigma >0$
%by using the method \texttt{L-BFGS-B} and set the lower limit of $\sigma $
%to 0 (the lower limit of $\mu $ is irrelevant and can be set to $-\infty $).
%Note that the data argument must not be of class \texttt{dataframe}, but of
%class \texttt{numeric}.
%
%\item Plot the histogram of income and add the fitted density to the plot.
%\end{enumerate}


\item Let $X\sim LN(\mu ,\sigma ^{2})$ and let $X_{1},\ldots ,X_{n}$ be a sample
drawn from $X$. The $X_{i}$ are not observable. Instead one can only observe%
\begin{equation*}
Y_{i}=\left\{
\begin{array}{ll}
X_{i} & \quad \text{if }X_{i}<c \\
c & \quad \text{if }X_{i}\geq c%
\end{array}%
\right.
\end{equation*}%
where $c$ is a known constant. The likelihood of $Y_{1},\ldots ,Y_{n}$ is
the product of all densities $f_{X}(y_{i}),$ for observations with $Y_{i}<c$%
, times the product of all probabilities that $Y_{i}=c$ for observations
with $Y_{i}=c$.

Let's consider the probability of $Y_i=c$:
\begin{equation*}
Pr(Y_i = c) = Pr(X_i \geq c) = 1 - Pr(X_i \leq c) = 1- F_X(c)
\end{equation*}
The likelihood function is now a mixture between the product of all densities $f_{X}(y_{i}),$ for observations with $Y_{i}<c$%
, times the product of all probabilities that $Y_{i}=c$ for observations with
$Y_{i}=c$:
\begin{equation*}
L(\mu,\sigma; y) = \prod_{i=1}^n\{ f_X(y_i;\mu,\sigma)\}^{\delta_i}\{1-F_X(c;\mu,\sigma)\}^{1-\delta_i}
\end{equation*}
with $\delta_i=1$ for exact observations and $\delta_i=0$ for a censored
observation. The loglikelihood is thus the sum of those two components (let
$n_1$ be the number of non-censored observations and $n_2$ the number of
censored observations, $n_1+n_2=n$):
\begin{equation*}
\log L(\mu,\sigma; y) = \sum_{i=1}^{n_1} \log f_X(y_i;\mu,\sigma) +
\sum_{i=1}^{n_2} \log(1-F_X(c;\mu,\sigma))
\end{equation*}


\begin{enumerate}
	\item Write an R function that computes the likelihood of $\mu $ and $\sigma
	^{2}$ given the observations $Y_{1},\ldots ,Y_{n}$ (and given $c$).
	
	\item Load the dataset \texttt{censoredln.csv}.
	
	\item Numerically maximize the likelihood function. The censoring value is $%
	c=12.$
	
	\item Compute the asymptotic covariance matrix of $\hat{\mu}$ and $\hat{%
		\sigma}^{2}$.
\end{enumerate}
\end{enumerate}
\newpage

\section{Dates and times}

\begin{enumerate}
\item Use the command \texttt{as.Date} to generate the following dates or
vectors of dates:

\begin{enumerate}
\item January 1st, 2012

\item November 9th, 1989

\item All Wednesdays in 2012

\item The first days of all months from January 2005 to December 2011.

\item The first days of all quarters from April 2000 to January 2012.

\item Use \texttt{difftime} to compute the number of days between your date of
birth and today.
\end{enumerate}

\item Use the command \texttt{strptime} to generate the following dates and
times or vectors of dates and times of class \texttt{POSIXlt}:

\begin{enumerate}
\item January 19th, 2012, 08:45:00

\item January 19th, 2012, 12:00:00

\item All minutes between the times in (a) and (b)

\item What happens if you add 1 to a \texttt{POSIXlt} object?
\end{enumerate}

\item Load the dataset \texttt{indices.csv} into a dataframe.

\begin{enumerate}
\item Use the command \texttt{strptime} to convert the first column of the
dataframe into a vector of class \texttt{POSIXlt}.

\item Plot the DAX index as a black line against the date vector.

\item Compute the daily returns of the DAX index,%
\begin{equation*}
r_{t}=\ln \frac{DAX_{t}}{DAX_{t-1}}
\end{equation*}%
and calculate the mean return for each weekday. Plot the mean
returns for each weekday as a bar chart.

\item Use the function \texttt{holiday} of the \texttt{timeDate} package to
identify the holidays at Deutsche B\"{o}rse. Remove the holiday returns and
redo the computations for the weekday returns.
\end{enumerate}
\end{enumerate}
\newpage


\section{Time series: Basics\label{tsbasics}}

Activate the \texttt{zoo} package.

\begin{enumerate}
\item Load the dataset \texttt{BIP.csv} into a dataframe. The first column
is the GDP in current prices, the second column is the price deflated
(chain) index of GDP with 2005=100.

\begin{enumerate}
\item The first observation is the first quarter of 2000, the last
observation is the third quarter of 2011. Create a \texttt{ts} time series
object of the GDP in current prices.

\item Print the time series.

\item Plot the time series.

\item Use the command \texttt{rollmean} to add a moving average of length $%
k=4$ to the plot.

\item Plot the time series of the differences between GDP and the rolling
mean.

\item Use the function \texttt{aggregate} with option \texttt{nfrequency=1}
to compute the time series of the annual GDP values.

\item Plot the differences of the logarithm of the annual GDP values (i.e.
the annual growth rates).
\end{enumerate}

\item Load the dataset \texttt{m3.csv} into a dataframe. It contains the
money aggregate M3 for the Euro area.\footnote{\texttt{%
https://stats.ecb.europa.eu/stats/download/bsi\_ma\_historical\_nsa\_dp/}%
\newline
\texttt{bsi\_ma\_historical\_nsa\_dp/bsi\_hist\_nsa\_u2\_2.pdf}} Remember to
set the option \texttt{as.is=TRUE} in the \texttt{read.csv} command.

\begin{enumerate}
\item Create a \texttt{zoo} object of the M3 time series.

\item Print the time series, setting the option \texttt{style} to each of
its three possible values, i.e. \texttt{horizontal}, \texttt{vertical}, and 
\texttt{plain}.

\item Plot the time series and add the rolling mean (with window width 12
months) in red.

\item Plot the deviation of the time series from its rolling mean.

\item Use the command \texttt{lag} to define the time series $M3_{t-12}$
(i.e. M3 in the same month of the previous year).

\item Plot $M3_{t}-M3_{t-12}$ and $\ln M3_{t}-\ln M3_{t-12}$.
\end{enumerate}

\item Convert the M3 time series to class \texttt{ts}.

\begin{enumerate}
\item Read \texttt{?decompose}. Decompose the time series into a trend, a
seasonal component, and a remainder term using the function \texttt{decompose%
}. Assign the function value to an object (which will be of class \texttt{%
decomposed.ts}), and plot this object.

\item Try both \texttt{type} options (i.e. \texttt{additive} and \texttt{%
multiplicative}). Which one would you prefer?
\end{enumerate}

\item Continue to use the M3 time series.

\begin{enumerate}
\item Execute \texttt{acf(diff(M3))} and interpret the plot.

\item Returns from the \texttt{acf} command are of class \texttt{acf}, see 
\texttt{?acf}. Extract the autocorrelations of \texttt{diff(M3)} as a vector.
\end{enumerate}

\item Consider the dataset \texttt{electricity.csv}. It contains hourly
electricity prices (in EUR/MWh) at the EEX.

\begin{enumerate}
\item The time format is YYYY-MM-DD, HH:00. Read the dataset into a
dataframe (using the option \texttt{as.is=TRUE}) and create a vector for the
date-time information.

\item Plot the time series of the electricity prices.

\item Use the \texttt{acf} command to compute and display the
autocorrelation function up to lag order 750.

\item Convert the price time series into an \texttt{ts} object with start
date 1 and end date 365 (and, of course, frequency 24). Use the command 
\texttt{aggregate} to calculate the average daily prices.
\end{enumerate}
\end{enumerate}
\newpage


\section{Time series: model estimation and unit roots}

Activate the packages \texttt{tseries}, \texttt{fGarch}, and \texttt{vars}.

\begin{enumerate}
\item Load the artificial dataset \texttt{ar1.csv}. It contains two
variables, the endogenous variable $Y$ and the exogenous variable $X$. The
linear relationship $Y_{t}=\alpha +\beta X_{t}$ is disturbed by a first
order autoregressive error term $u_{t}$ with%
\begin{equation*}
u_{t}=\rho u_{t-1}+\varepsilon _{t}
\end{equation*}%
where $\varepsilon _{t}\sim N(0,\sigma ^{2})$ and $|\rho |<1$.

\begin{enumerate}
\item Run an OLS regression of $Y$ on $X$ and put the residuals into a
vector \texttt{uhat}.

\item Use the command \texttt{ar} to estimate the autoregression coeffcient $%
\rho $ and the variance $\sigma ^{2}$.

\item Extract the standard error of $\hat{\rho}$. Is $\hat{\rho}$
significantly positive?
\end{enumerate}

\item Load the dataset \texttt{BIP.csv} and convert the first column (i.e.
GDP in current prices) to class \texttt{ts}.\footnote{%
See exercise \ref{tsbasics}.1.}

\begin{enumerate}
\item Use the command \texttt{decompose} to calculate the trend, the
saisonal pattern and the remainder term of the time series.

\item Fit an $AR(1)$ process to the remainder term (\texttt{\$random}). In
order to remove the missing values set the option \texttt{na.action=na.omit}.

\item Test if the first order autocorrelation coefficient is significantly
different from zero.

\item Now use the command \texttt{arima} to fit an $AR(1)$ model to the
remainder term.

\item Fit an $MA(1)$ model to the remainder term.
\end{enumerate}

\item Load the dataset \texttt{investment.csv}. It contains the gross
machinery investment (Bruttoausr\"{u}stungsinvestitionen) in current prices
(in billion Euros).

\begin{enumerate}
\item Convert the investment time series to class \texttt{ts} and plot it in
levels and percentage changes (differences of the logs). Ignore the
structural break in 1990 due to Germany's unification.

\item Fit $ARIMA(p,1,q)$ models to the time series with $p,q\in \{0,1,2\}$.
Find the lowest $AIC$ value of the 9 models.

\item Perform an $ADF$ test. Does investment exhibit a unit root?

\item Perform an $ADF$ test for the differenced time series of investment.
Do the differences have a unit root?
\end{enumerate}

\item Load the dataset \texttt{indices.csv} and compute the returns of the
DAX index as%
\begin{equation*}
r_{t}^{DAX}=\ln \frac{DAX_{t}}{DAX_{t-1}}.
\end{equation*}

\begin{enumerate}
\item Fit a $GARCH(1,1)$ model to the returns by \texttt{garch}.

\item Fit a $GARCH(1,1)$ model by \texttt{garchFit} and compare your results
with (a).

\item Add an $AR(2)$ mean equation.
\end{enumerate}

\item Consider the bivariate $VAR(1)$ process%
\begin{equation*}
y_{t}=\left[ 
\begin{array}{cc}
0.9 & 0 \\ 
0.5 & 0.5%
\end{array}%
\right] y_{t-1}+\varepsilon _{t},\qquad \varepsilon _{t}\sim N\left( \left[ 
\begin{array}{c}
0 \\ 
0%
\end{array}%
\right] ,\left[ 
\begin{array}{cc}
1 & 0.3 \\ 
0.3 & 1%
\end{array}%
\right] \right)
\end{equation*}%
where $y_{t}=(y_{1t},y_{2t})^{\prime }$, see A. Staszewska-Bystrova (2011),
Bootstrap Prediction Bands for Forecast Paths from Vector Autoregressive
Models, \emph{Journal of Forecasting}, 30: 721-735.

\begin{enumerate}
\item Write an R program that performs the following steps.

\begin{itemize}
\item Set $R=1000.$

\item Initiate an empty matrix $Z$ of dimension $R\times 4$.

\item For $r=1,\ldots ,R$: Set $y_{1}=(0,0)$; simulate $y_{t}$ for $%
t=2,\ldots ,200$; drop the first 100 observations; use the \texttt{VAR}
command of the \texttt{vars} package to estimate the four coefficients of
the autoregression matrix; save the coefficients in row $r$ of matrix $Z$.
\end{itemize}

\item Use the \texttt{apply} command to compute the mean estimates of the
four coefficients. Are the VAR estimators biased?

\item Plot the histograms of the four coefficients in a $2\times 2$ plot.
\end{enumerate}


\end{enumerate}
\newpage


\section{Graphics}
The following exercises rely on a Harvard University tutorial called "Introduction to R Graphics with ggplot2". The exercises show you how to investigate a dataset about housing prices visually using \texttt{ggplot2}. 
\begin{enumerate}
\item Load the dataset \texttt{landdata-states.csv} and create a histogramm of Home.Value using \texttt{aes} and \texttt{geom\_histogram}.
\item Redo the first exercise using an appropriate binwidth using \texttt{binwidth}.
\item Plot two countries of your choice over time  with two different colours using \texttt{geom\_point}, \texttt{subset} and \texttt{color}.
\item Load the package ggThemeAssist and 
polish your last graphic \\ using \texttt{ggThemeAssistGadget(ggplot2-object)}.
\item Make a scatterplot of Land.Value in logs on the x-axis and Strcuture.Cost on the y-axis for the first quarter of 2001 using \texttt{geom\_point} and \texttt{subset}. (We refer to this as the basis plot in the following.)
\item Add a fitted linear regression line to the basis plot and color the points with respect to Home.Value using \texttt{geom\_smooth} and \texttt{color}.
\item Use a fitted polynomial instead of the regression line in the last plot.
\item Use the unique variable names in State as y-values in the basis plot using \texttt{geom\_text}
\item Use different shapes due to the variable region with \texttt{shape} and different colours due to the variable Home.Value with \texttt{color} for the last plot.
\item Calculate the mean of Home.Value for every state using \texttt{aggregate} and plot your new data.frame using \texttt{geom\_bar} and \texttt{stat = "identity"}.
\item Plot State on the x-axis and Home.Price.Index on the y-axis using \texttt{geom\_point}. Use Date as color to distinguish the plot with respect to the time.
\item Use \texttt{alpha} and \texttt{size} to make the last plot clearer.
\item Use \texttt{scale\_color\_gradient} to change the color of the last plot.
\item Plot the Home.Value over time for every state in a different colour.
\item Divide the last plot in multiple plots each for every state using \texttt{facet\_wrap}.
\end{enumerate}

\end{document}
